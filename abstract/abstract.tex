% !TeX encoding = UTF-8

The subject matter of this talk is the first Grigorchuk group in context of
Burnside's problems. Since this group, which was introduced by Rostislav
Grigorchuk (Ukr.: Григорчук) in 1980, is a subgroup of the automorphism group
of a full binary tree, in the first part of this talk I will introduce some
graph theoretical notions. The second part shows that the first Grigorchuk
group poses a counterexample to the \emph{unbounded Burnside problem}, which
was first stated in a paper by William Burnside in 1902, and asks whether each
finitely generated periodic group is finite. Burnside's question had
significant impact on the whole field of group theory throughout the
20\textsuperscript{th} century. In the last part I will talk about the growth
of the first Grigorchuk group with respect to the word length. I will sketch a
proof that the orders of the automorphisms in this group cannot be bound
uniformly. As a consequence, the first Grigorchuk group does not pose a
counterexample to the stronger \emph{bounded Burnside problem}.

Depending on the preference of the audience this talk will be given in English
or German.

