% !TeX spellcheck = en_GB
% !TEX TS-program = xelatex

\documentclass[compress]{beamer}
\usetheme[progressbar=frametitle, block=fill]{metropolis}

\setsansfont[Numbers=Lining]{EB Garamond}
\setmonofont[Scale=MatchLowercase]{Monaco}

\usepackage{polyglossia}

\usepackage{booktabs}
\usepackage[scale=2]{ccicons}



\usepackage{amsmath,amsfonts,amssymb,enumerate,hyperref}
\usepackage{pgf}
\usepackage{tikz}
\usetikzlibrary{arrows, backgrounds}

\usepackage{xcolor}
\definecolor{verysofblue}{HTML}{88CCEE}
\definecolor{darkblue}{HTML}{332288}
\definecolor{darkmoderatecyan}{HTML}{44AA99}
\definecolor{darkcyan}{HTML}{117777}
\definecolor{softyellow}{HTML}{DDCC77}
\definecolor{darkmoderatemagenta}{HTML}{AA4499}
\definecolor{darkpink}{HTML}{882255}
\definecolor{moderatered}{HTML}{CC6677}
\definecolor{darkred}{HTML}{AE1C3E}
\definecolor{mDarkTeal}{HTML}{23373b}
%\definecolor{mLightBrown}{HTML}{EB811B}

\colorlet{cLines}{mDarkTeal}
\colorlet{cHighlight}{darkred}
\colorlet{cLeft}{moderatered}
\colorlet{cRight}{verysofblue}



\setbeamercolor{progress bar}{fg=cHighlight}

\tikzset{
 onslide/.code args={<#1>#2}{%
  \only<#1>{\pgfkeysalso{#2}} % \pgfkeysalso doesn't change the path
 }
}

\tikzset{
 v_l/.style = {fill=cLines, text=white},
 st_l/.style = {draw=cLeft},
 st_r/.style = {draw=cRight},
 highlight/.style={draw=cHighlight, line width=3pt},
 hlarrow/.style={<->, shorten <=1pt, shorten >=1pt, draw=cHighlight, ultra thick}
}

%\usepackage{pgfpages}
%\pgfpagesuselayout{2 on 1}[a4paper,border shrink=5mm]

\author{Tim B.\ Herbstrith}
\title{The First Grigorchuk Group}
\subtitle{in Context of Burnside's Problems}
%\institute{1063642}
\date{Vienna, 28\textsuperscript{th} April 2017}
%\logo{\pgfimage[width=1.5cm,height=1.5cm]{UniWienSiegel}}

%\setbeamercovered{transparent}
%\setbeamercolor{block title}{use=structure,fg=red!75!black, bg=white!90!black}
%\setbeamertemplate{itemize item}[circle]
%\setbeamertemplate{enumerate item}[default]
%\setbeamercolor{itemize item}{fg=black}
%\setbeamercolor{enumerate item}{fg=black}
%\setbeamertemplate{navigation symbols}{}
%\setbeamertemplate{blocks}[default]

% Tightlist for pandoc
\providecommand{\tightlist}{%
  \setlength{\itemsep}{0pt}\setlength{\parskip}{0pt}}

\renewcommand{\arraystretch}{1.5}

\DeclareMathOperator{\N}{\mathbb{N}}
\DeclareMathOperator{\Z}{\mathbb{Z}}
\DeclareMathOperator{\Q}{\mathbb{Q}}
\DeclareMathOperator{\R}{\mathbb{R}}
\DeclareMathOperator{\F}{\mathbb{F}}
\DeclareMathOperator{\Cay}{Cay}
\DeclareMathOperator{\id}{id}
\DeclareMathOperator{\rk}{rk}
\DeclareMathOperator{\ls}{L(S)}
\DeclareMathOperator{\kernel}{ker}
\DeclareMathOperator{\End}{\mathrm{End}}
\DeclareMathOperator{\aut}{\mathrm{Aut}}

\newcommand*{\seq}[2][n]{#2_1, \ldots, #2_{#1}}

\newtheorem{theo}{Satz}[section]
\newtheorem{lem}[theo]{Lemma}
\newtheorem{prop}[theo]{Proposition}
\newtheorem{cor}[theo]{Corollary}
\theoremstyle{definition}
\newtheorem{defin}[theo]{Definition}
\newtheorem{exam}[theo]{Example}
\theoremstyle{remark}
\newtheorem*{rem}{Remark}

\usepackage{nicefrac}

\begin{document}
  \begin{frame}
    \maketitle
  \end{frame}
  
  \include{index}
%\section{Burnside's Problems}
%
%\subsection{Problem Statement}
%
%\begin{frame}{Periodic Groups}
%    \begin{defin}
%        A group $G$ is called \emph{periodic} if for each element $a \in G$ there exists an integer $ n > 0 $ such that $a^n = e$.
%    \end{defin}
%
%    \begin{exam}<2>
%        \begin{itemize}
%            \item Every finite group is periodic.
%            \item The direct sum of cyclic groups
%            \[\bigoplus_{i=2}^\infty \nicefrac \Z{i\Z}\]
%            is periodic.
%        \end{itemize}
%    \end{exam}
%\end{frame}
%
%\begin{frame}{Periodic Groups}
%\begin{defin}
%    A group $G$ is called \emph{periodic} if for each group element $a \in G$ there exists an integer $n_a > 0$ such that $a^{n_a} = 1$. 
%\end{defin}
%
%\begin{exam}<2>
%\begin{itemize}
%\item Jede endliche Gruppe ist periodisch.
%\item Die direkte Summe
%\begin{equation*}
%\bigoplus_{i=2}^\infty\Z_i
%\end{equation*}
%ist versehen mit komponentenweiser Addition eine periodische Gruppe.
%\end{itemize}
%\end{exam}
%\end{frame}
%
%\begin{frame}
%\begin{block}{Unbounded Burnside's Problem (1902)}
%Ist jede endlich erzeugte, periodische Gruppe endlich?
%\end{block}
%
%\begin{exam}[Positive Resultate]<2>
%\begin{itemize}
%\item Jede endlich erzeugte, periodische, abelsche Gruppe ist endlich. (Hauptsatz \"uber endlich erzeugte abelsche Gruppen)
%\item Jede endlich erzeugte, periodische Untergruppe einer linearen Gruppe $\mathrm{GL}_n(\mathbb{K})$ ist endlich. (Schur 1911)
%\end{itemize}
%\end{exam}
%\end{frame}
%
%\subsection{Grigorchuks Gegenbeispiel}
%\subsubsection{Elementare Eigenschaften und Definitionen}
%
%\begin{frame}{Graphen}
%\begin{defin}
%\begin{enumerate}
%\item Ein \emph{Graph} ist ein Tupel $G=(V,E)$ bestehend aus der Knotenmenge $V$ und der Kantenmenge $E\subseteq V\times V$, wobei aus $(v_1,v_2)\in E$ stets $(v_2,v_1)\in E$ folgt.
%\item Zwei Knoten $v_1$ und $v_2$ aus $V$ hei\ss{}en \emph{verbunden}, wenn $(v_1,v_2)\in E$.
%\item $V(G)$ bezeichnet die Knotenmenge von $G$.
%\end{enumerate}
%\end{defin}
%\end{frame}
%
%\begin{frame}{Beispiel eines Graphen}
%\begin{minipage}{0.3\textwidth}
%\begin{center}
%\begin{tikzpicture}[level distance=10mm]
%\tikzstyle{every node}=[circle,draw=black,thick,font=\small,text=black,inner sep=1pt, minimum size=5mm]
%\foreach \name/\angle/\text in {P-4/234/4, P-5/162/5, 
%                                  P-1/90/1, P-2/18/2, P-3/-54/3}
%    \node (\name) at (\angle:1cm) {$\text$};
%\foreach \from/\to in {1/2,2/3,3/4,4/5,5/1,2/4,3/5,5/2}
%     \draw[draw=black,thick] (P-\from) -- (P-\to);
%\end{tikzpicture}
%\end{center}
%\end{minipage}
%\begin{minipage}{0.6\textwidth}
%F\"ur den linken Graphen $G=(V,E)$ gilt
%\begin{align*}
%V&=\lbrace1,2,3,4,5\rbrace \text{ und}\\
%E&=\lbrace (1,2),(1,5),\\
%&(2,1),(2,3),(2,4),(2,5),\\
%&(3,2),(3,4),(3,5),\\
%&(4,2),(4,3),(4,5),\\
%&(5,1),(5,2),(5,3),(5,4)\rbrace
%\end{align*}
%\end{minipage}
%\end{frame}
%
%\begin{frame}{Automorphismengruppe eines Graphs}
%\begin{defin}
%Seien $G_1=(V_1,E_1)$ und $G_2=(V_2,E_2)$ zwei Graphen. Eine bijektive Abbildung $f\colon V_1\to V_2$ hei\ss{}t \textit{Graphenisomorphismus}, wenn $(v_1,v_2)\in E_1$ genau dann, wenn $(f(v_1),f(v_2))\in E_2$.\\
%Gilt $G_1=G_2$, so wird $f$ ein \textit{Graphenautomorphismus} genannt.
%\end{defin}
%
%\begin{theo}<2>
%Die Menge $\aut(G)$ aller Graphenautomorphismen eines Graphen $G$ bildet bez\"uglich Verkn\"upfung von Abbildungen eine Gruppe. 
%\end{theo}
%\end{frame}
%
%\begin{frame}{Bin\"arb\"aume}
%\begin{center}
%\begin{tikzpicture}[level distance=10mm]
%  \tikzstyle{edge from parent}=[draw=black,very thick]
%  \tikzstyle{every node}=[circle,draw=black,very thick,font=\small,text=black,inner sep=1pt, minimum size=5mm]
%  \tikzstyle{level 1}=[sibling distance=60mm]
%  \tikzstyle{level 2}=[sibling distance=30mm]
%  \tikzstyle{level 3}=[sibling distance=15mm]
%  \tikzstyle{level 4}=[sibling distance=7.5mm]
%  \tikzstyle{level 5}=[sibling distance=3.75mm, level distance=5mm]
%  
%  
%  \node [fill=black!30] (root) {$\emptyset$}
%  	child {node [v_l,onslide={<2> highlight}] {-1} edge from parent [onslide={<2> highlight}]
%  		child {node [v_l,onslide={<2> highlight}] {-1} edge from parent [onslide={<2> highlight}]
%  			child {node [v_l] {-1}
%  				child {node [v_l] {-1} child foreach \a in {0,1}{coordinate edge from parent[dotted]}}
%  				child {node {1} child foreach \a in {0,1}{coordinate edge from parent[dotted]}}
%  				}
%  			child {node [onslide={<2> highlight}, onslide={<3> highlight, fill=red, text=white}] {1}
%  				edge from parent [onslide={<2> highlight}]
%  				child {node [v_l, onslide={<4> highlight, fill=red, text=white}] {-1} child foreach \a in {0,1}
%  				{coordinate edge from parent[dotted]}}
%  				child {node [onslide={<2-> highlight, fill=red, text=white}] {1} edge from parent [onslide={<2,3> highlight}]
%  					child foreach \a in {0,1}{coordinate edge from parent[dotted, draw=black]}}
%  				}
%  			}
%  		child {node {1}
%  			child {node [v_l] {-1}
%  				child {node [v_l] {-1} child foreach \a in {0,1}{coordinate edge from parent[dotted]}}
%  				child {node {1} child foreach \a in {0,1}{coordinate edge from parent[dotted]}}
%  				}
%  			child {node {1}
%  				child {node [v_l] {-1} child foreach \a in {0,1}{coordinate edge from parent[dotted]}}
%  				child {node {1}  child foreach \a in {0,1}{coordinate edge from parent[dotted]}}
%  				}
%  			}
%  		}
%	child {node {1}
%  		child {node [v_l] {-1}
%  			child {node [v_l] {-1}
%  				child {node [v_l] {-1} child foreach \a in {0,1}{coordinate edge from parent[dotted]}}
%  				child {node {1} child foreach \a in {0,1}{coordinate edge from parent[dotted]}}
%  				}
%  			child {node {1}
%  				child {node [v_l] {-1} child foreach \a in {0,1}{coordinate edge from parent[dotted]}}
%  				child {node {1} child foreach \a in {0,1}{coordinate edge from parent[dotted]}}
%  				}
%  			}
%  		child {node {1}
%  			child {node [v_l] {-1}
%  				child {node [v_l] {-1} child foreach \a in {0,1}{coordinate edge from parent[dotted]}}
%  				child {node {1} child foreach \a in {0,1}{coordinate edge from parent[dotted]}}
%  				}
%  			child {node {1}
%  				child {node [v_l] {-1} child foreach \a in {0,1}{coordinate edge from parent[dotted]}}
%  				child {node {1} child foreach \a in {0,1}{coordinate edge from parent[dotted]}}
%  				}
%  			}
%  		}
%  	;		
%\end{tikzpicture}
%\end{center}
%\begin{overlayarea}{\textwidth}{3em}
%\only<2>{Den rot hinterlegten Knoten identifizieren wir mit der Folge $(-1,-1,1,1)$.}
%\only<3>{Die Knoten $(-1,-1,1)$ und $(-1,-1,1,1)$ sind miteinander verbunden.}
%\only<4>{Aber $(-1,-1,1,-1)$ und $(-1,-1,1,1)$ sind nicht verbunden.}
%\end{overlayarea}
%\end{frame}
%
%\begin{frame}{Bin\"arb\"aume}
%\begin{defin}
%Ein Bin\"arbaum $T^{(2)}=(V,E)$ ist ein Graph dessen Knotenmenge
%\begin{equation*}
%V:=\lbrace (b_i)_{i=1}^n\mid n\in \Z, n\geq 0, b_i\in\lbrace -1,1\rbrace\text{ f\"ur }1\leq i\leq n\rbrace
%\end{equation*}
%der Menge aller endlichen bin\"aren Folgen entspricht und dessen Kantenmenge $E$ gegeben ist durch
%\begin{equation*}
%E:=\lbrace ((b_i)_{i=1}^n,(b'_i)_{i=1}^m)\mid  ((b_i)_{i=1}^n,(b'_i)_{i=1}^m)\in E' \vee ((b'_i)_{i=1}^m,(b_i)_{i=1}^n)\in E'\rbrace,
%\end{equation*}
%wobei
%\begin{equation*}
%E':=\lbrace ((b_i)_{i=1}^n,(b'_i)_{i=1}^m)\in V\times V\mid m=n+1\text{ und } b_i=b'_i\text{ f\"ur } 1\leq i\leq n\rbrace.
%\end{equation*}
%
%\end{defin}
%\end{frame}
%
%\begin{frame}{Eigenschaften von $\aut(T^{(2)})$}
%\begin{theo}
%F\"ur den Bin\"arbaum $T^{(2)}=(V,E)$ gelten:
%\begin{enumerate}
%\item Jeder Automorphismus $\varphi\in\aut{(T^{(2)})}$ h\"alt die Wurzel $(\emptyset)$ von $T^{(2)}$ fest, d.h. $\varphi((\emptyset))=(\emptyset)$.
%\item  Jeder Automorphismus $\varphi\in\aut{(T^{(2)})}$ permutiert nur Knoten in einer Ebene, d.h. f\"ur alle $n\geq 1$ und alle $(b_i)_{i=1}^n\in V$ gilt
%\begin{equation*}
%\varphi((b_i)_{i=1}^n)=(c_i)_{i=1}^n
%\end{equation*}
%f\"ur gewisse $c_1,\ldots,c_n\in \lbrace -1,1\rbrace$.
%\end{enumerate}
%\end{theo}
%\end{frame}
%
%\begin{frame}
%\begin{lem}
% Die Teilmenge
%\begin{align*}
%St(m):=\lbrace \varphi\in\aut(T^{(2)})&\mid \varphi((b_i)_{i=1}^n)=(b_i)_{i=1}^n,\\
%& \forall (b_i)_{i=1}^n\in V(T^{(2)}) \text{ mit } n\leq m\rbrace,
%\end{align*}
%d.h. die Menge aller Automorphismen, die Folgen mit einer L\"ange kleiner gleich $m$ fixieren, bildet eine Untergruppe.
%\end{lem}
%\begin{defin}<2>
%Die im vorherigen Lemma eingef\"uhrte Gruppe wird \emph{$m$-te Stabilisatorgruppe} genannt.
%\end{defin}
%\end{frame}
%
%\begin{frame}{Teilb\"aume $T_{(b_i)_{i=1}^n}^{(2)}$}
%\begin{center}
%\begin{tikzpicture}[level distance=10mm]
%  \tikzstyle{edge from parent}=[draw,very thick]
%  \tikzstyle{every node}=[circle,draw,very thick,font=\small,text=mDarkTeal,inner sep=1pt, minimum size=5mm]
%  \tikzstyle{level 1}=[sibling distance=60mm]
%  \tikzstyle{level 2}=[sibling distance=30mm]
%  \tikzstyle{level 3}=[sibling distance=15mm]
%  \tikzstyle{level 4}=[sibling distance=7.5mm]
%  \tikzstyle{level 5}=[sibling distance=3.75mm, level distance=5mm]
%  
%  \node [fill=black!30] (root) {$\emptyset$}
%  	child {node [v_l,onslide={<2> highlight}] {-1} edge from parent [onslide={<2> highlight}]
%  		child[] {node [v_l,onslide={<2> highlight}, onslide={<1,3> st_l}] {-1}
%  		 edge from parent [onslide={<2> highlight}]
%  			child[st_l] {node [v_l] {-1} edge from parent [draw=moderatered]
%  				child {node [v_l] {-1} child foreach \a in {0,1}{coordinate edge from parent[dotted]}}
%  				child {node {1} child foreach \a in {0,1}{coordinate edge from parent[dotted]}}
%  				}
%  			child[onslide={<1-3> st_l}] {node[onslide={<2,3> highlight}] {1}
%  			  edge from parent [onslide={<2,3> highlight}]
%  				child {node[onslide={<2,3> draw=moderatered}] [v_l] {-1} edge from parent [onslide={<2,3> draw=moderatered}]
%  				  child foreach \a in {0,1}{coordinate edge from parent[dotted]}}
%  				child {node[onslide={<2,3> fill=darkpink, highlight}] {1} edge from parent [onslide={<2,3> highlight}]
%  				  child foreach \a in {0,1}{coordinate edge from parent[dotted, onslide={<2,3> draw=moderatered}]}}
%  				}
%  			}
%  		child[] {node[onslide={<2> draw=mDarkTeal}] {1} edge from parent [onslide={<2> draw=black}]
%  			child {node [v_l] {-1}
%  				child {node [v_l] {-1} child foreach \a in {0,1}{coordinate edge from parent[dotted]}}
%  				child {node {1} child foreach \a in {0,1}{coordinate edge from parent[dotted]}}
%  				}
%  			child {node {1}
%  				child {node [v_l] {-1} child foreach \a in {0,1}{coordinate edge from parent[dotted]}}
%  				child {node {1}  child foreach \a in {0,1}{coordinate edge from parent[dotted]}}
%  				}
%  			}
%  		}
%	child {node[] {1}
%  		child[] {node [v_l] {-1}
%  			child {node [v_l] {-1}
%  				child {node [v_l] {-1} child foreach \a in {0,1}{coordinate edge from parent[dotted]}}
%  				child {node {1} child foreach \a in {0,1}{coordinate edge from parent[dotted]}}
%  				}
%  			child {node {1}
%  				child {node [v_l] {-1} child foreach \a in {0,1}{coordinate edge from parent[dotted]}}
%  				child {node {1} child foreach \a in {0,1}{coordinate edge from parent[dotted]}}
%  				}
%  			}
%  		child[] {node {1}
%  			child {node [v_l] {-1}
%  				child {node [v_l] {-1} child foreach \a in {0,1}{coordinate edge from parent[dotted]}}
%  				child {node {1} child foreach \a in {0,1}{coordinate edge from parent[dotted]}}
%  				}
%  			child {node {1}
%  				child {node [v_l] {-1} child foreach \a in {0,1}{coordinate edge from parent[dotted]}}
%  				child {node {1} child foreach \a in {0,1}{coordinate edge from parent[dotted]}}
%  				}
%  			}
%  		}
%  	;
%  	
%  	\node[draw=none, font=\large, text=moderatered]  [above of=root-1-1] {$T_{(-1,-1)}^{(2)}$};
%\end{tikzpicture}
%\end{center}
%
%\begin{overlayarea}{\textwidth}{3em}
%\only<2,3>{Den rot hinterlegten Knoten identifizieren wir mit der Folge $(-1,-1,1,1)\in V(T^{(2)})$}\only<2>{.}
%\only<3>{und mit $(1,1)\in V(T_{(-1,-1)}^{(2)})$.}
%\end{overlayarea}
%\end{frame}
%
%\begin{frame}
%\begin{defin}
%Mit $T_{(b_i)_{i=1}^n}^{(2)}$ sei der Teilbaum von $T^{(2)}$, dessen Knotenmenge der Menge aller Fortsetzungen der Folge $(b_i)_{i=1}^n$ entspricht, bezeichnet. Wir k\"onnen $T_{(b_i)_{i=1}^n}^{(2)}$ selbst als Bin\"arbaum auf{}fassen.
%\end{defin} 
%\begin{lem}<2>
%Die Abbildung $\psi\colon St(1)\to \aut(T^{(2)})\times\aut(T^{(2)})$ definiert durch
%\begin{equation*}
%\psi(\varphi)=(\varphi\big|_{T_{(-1)}^{(2)}},\varphi\big|_{T_{(1)}^{(2)}})
%\end{equation*}
%ist ein Gruppenisomorphismus.
%\end{lem}
%\end{frame}
%
%\begin{frame}{Die Abbildung $\psi$}
%\begin{center}
%\begin{tikzpicture}[level distance=10mm]
%  \tikzstyle{edge from parent}=[very thick, draw]
%  \tikzstyle{every node}=[circle,draw,very thick,font=\small,text=black,inner sep=1pt, minimum size=5mm]
%  \tikzstyle{level 1}=[sibling distance=60mm]
%  \tikzstyle{level 2}=[sibling distance=30mm]
%  \tikzstyle{level 3}=[sibling distance=15mm]
%  \tikzstyle{level 4}=[sibling distance=7.5mm]
%  \tikzstyle{level 5}=[sibling distance=3.75mm, level distance=5mm]
%  
%  \node [fill=black!30] (root) {$\emptyset$}
%  	child {node [v_l,st_l] {-1}
%  		child[st_l] {node[v_l] {-1}
%  			child {node [v_l] {-1}
%  				child {node [v_l] {-1} child foreach \a in {0,1}{coordinate edge from parent[dotted]}}
%  				child {node {1} child foreach \a in {0,1}{coordinate edge from parent[dotted]}}
%  				}
%  			child {node {1}
%  				child {node [v_l] {-1} child foreach \a in {0,1}{coordinate edge from parent[dotted]}}
%  				child {node {1} child foreach \a in {0,1}{coordinate edge from parent[dotted]}}
%  				}
%  			}
%  		child[st_l] {node {1}
%  			child {node [v_l] {-1}
%  				child {node [v_l] {-1} child foreach \a in {0,1}{coordinate edge from parent[dotted]}}
%  				child {node {1} child foreach \a in {0,1}{coordinate edge from parent[dotted]}}
%  				}
%  			child {node {1}
%  				child {node [v_l] {-1} child foreach \a in {0,1}{coordinate edge from parent[dotted]}}
%  				child {node {1} child foreach \a in {0,1}{coordinate edge from parent[dotted]}}
%  				}
%  			}
%  		}
%	child {node[st_r] {1}
%  		child[st_r] {node[v_l] {-1}
%  			child {node [v_l] {-1}
%  				child {node [v_l] {-1} child foreach \a in {0,1}{coordinate edge from parent[dotted]}}
%  				child {node {1} child foreach \a in {0,1}{coordinate edge from parent[dotted]}}
%  				}
%  			child {node {1}
%  				child {node [v_l] {-1} child foreach \a in {0,1}{coordinate edge from parent[dotted]}}
%  				child {node {1} child foreach \a in {0,1}{coordinate edge from parent[dotted]}}
%  				}
%  			}
%  		child[st_r] {node {1}
%  			child {node [v_l] {-1}
%  				child {node [v_l] {-1} child foreach \a in {0,1}{coordinate edge from parent[dotted]}}
%  				child {node {1} child foreach \a in {0,1}{coordinate edge from parent[dotted]}}
%  				}
%  			child {node {1}
%  				child {node [v_l] {-1} child foreach \a in {0,1}{coordinate edge from parent[dotted]}}
%  				child {node {1} child foreach \a in {0,1}{coordinate edge from parent[dotted]}}
%  				}
%  			}
%  		}
%  	;
%
%  	\node[draw=none, font=\large, text=blue]  [above of=root-1] {$T_{(-1)}^{(2)}$};
%  	\node[draw=none, font=\large, text=green!60!black]  [above of=root-2] {$T_{(1)}^{(2)}$};
%\end{tikzpicture}
%\end{center}
%
%
%$\psi$ beschreibt die Wirkung einer Abbildung auf den linken und den rechten Teilbaum.
%
%\end{frame}
%
%\begin{frame}{1. Grigorchk-Gruppe}
%\begin{defin}
%\begin{enumerate}
%\item Es seien $a,b,c$ und $d$ aus $\aut(T^{(2)})$ definiert durch
%\begin{align*}
%a(b_1,b_2,\ldots,b_n)&:= (-b_1,b_2,\ldots,b_n)\\
%b&:=\psi^{-1}(a,c)\\
%c&:=\psi^{-1}(a,d) \text{ und}\\
%d&:=\psi^{-1}(id,b).
%\end{align*}
%\item Die \textit{1. Grigorchuk-Gruppe} $\Gamma$ ist die von $a,b,c$ und $d$ erzeugte Untergruppe von $\aut(T^{(2)})$, d.h.
%\begin{equation*}
%\Gamma:=\langle a,b,c,d\rangle.
%\end{equation*}
%\end{enumerate}
%\end{defin}
%\end{frame}
%
%\begin{frame}
%Es gilt $b:=\psi^{-1}(a,c), c:=\psi^{-1}(a,d)$ und $d:=\psi^{-1}(id,b)$. 
%\begin{center}
%\begin{tikzpicture}[>=angle 45, level distance=10mm]
%  \tikzstyle{edge from parent}=[very thick, draw]
%  \tikzstyle{every node}=[circle,draw,very thick,font=\small,text=black,inner sep=1pt, minimum size=5mm]
%  \tikzstyle{level 1}=[sibling distance=60mm]
%  \tikzstyle{level 2}=[sibling distance=30mm]
%  \tikzstyle{level 3}=[sibling distance=15mm]
%  \tikzstyle{level 4}=[sibling distance=7.5mm]
%  \tikzstyle{level 5}=[sibling distance=3.75mm, level distance=5mm]
%  
%  \node [fill=black!30] (root) {$\emptyset$}
%  	child {node [onslide={<1,2,4-> v_l},onslide={<5,10,15> st_l}] {$\only<1,2,4->{\text{-1}}\only<3>{\text{1}}$}
%  		child[onslide={<5,10,15> st_l}]
%  		 {node [onslide={<1-5,9-10,14-> v_l}] {$\only<1-5,9-10,14->{\text{-1}}\only<6-8,11-13>{\text{1}}$}
%  			child {node [v_l] {-1}
%  				child {node [v_l] {-1} child foreach \a in {0,1}{coordinate edge from parent[dotted]}}
%  				child {node {1} child foreach \a in {0,1}{coordinate edge from parent[dotted]}}
%  				}
%  			child {node {1}
%  				child {node [v_l] {-1} child foreach \a in {0,1}{coordinate edge from parent[dotted]}}
%  				child {node {1}  child foreach \a in {0,1}{coordinate edge from parent[dotted]}}
%  				}
%  			}
%  		child[onslide={<5,10,15> st_l}]
%  		 {node [onslide={<6-8,11-13> v_l}] {$\only<6-8,11-13>{\text{-1}}\only<1-5,9-10,14->{\text{1}}$}
%  			child {node [v_l] {-1}
%  				child {node [v_l] {-1} child foreach \a in {0,1}{coordinate edge from parent[dotted]}}
%  				child {node {1} child foreach \a in {0,1}{coordinate edge from parent[dotted]}}
%  				}
%  			child {node {1}
%  				child {node [v_l] {-1} child foreach \a in {0,1}{coordinate edge from parent[dotted]}}
%  				child {node {1}  child foreach \a in {0,1}{coordinate edge from parent[dotted]}}
%  				}
%  			}
%  		}
%	child {node [onslide={<3> v_l},onslide={<5,10,15> st_r}] {$\only<3>{\text{-1}}\only<1,2,4->{\text{1}}$}
%  		child[onslide={<5,10,15> st_r}] {node [v_l,onslide={<6,11,16> st_l}] {-1}
%  			child[onslide={<6,11,16> st_l}]
%  			 {node [onslide={<1-6,9-16> v_l}] {$\only<1-6,9-16>{\text{-1}}\only<7-8,17->{\text{1}}$}
%  				child {node [v_l] {-1} child foreach \a in {0,1}{coordinate edge from parent[dotted]}}
%  				child {node {1} child foreach \a in {0,1}{coordinate edge from parent[dotted]}}
%  				}
%  			child[onslide={<6,11,16> st_l}]
%  			 {node [onslide={<7-8,17-> v_l}] {$\only<7-8,17->{\text{-1}}\only<1-6,9-16>{\text{1}}$}
%  				child {node [v_l] {-1} child foreach \a in {0,1}{coordinate edge from parent[dotted]}}
%  				child {node {1} child foreach \a in {0,1}{coordinate edge from parent[dotted]}}
%  				}
%  			}
%  		child[onslide={<5,10,15> st_r}] {node [onslide={<6,11,16> st_r}] {1}
%  			child[onslide={<6,11,16> st_r}] {node [v_l, onslide={<7,12,17> st_l}] {-1}
%  				child[onslide={<7,12,17> st_l}]
%  				 {node [onslide={<-12,14-17> v_l}]
%  				  {$\only<-12,14-17>{\text{-1}}\only<13,18>{\text{1}}$}
%  				   child foreach \a in {0,1}{coordinate edge from parent[dotted]}}
%  				child[onslide={<7,12,17> st_l}]
%  				 {node[fill=red] {$\only<13,18>{\text{-1}}\only<-12,14-17>{\text{1}}$}
%  				  child foreach \a in {0,1}{coordinate edge from parent[dotted]}}
%  				}
%  			child[onslide={<6,11,16> st_r}] {node [onslide={<7,12,17> st_r}] {1}
%  				child[onslide={<7,12,17> st_r}]
%  				 {node [v_l] {-1} child foreach \a in {0,1}{coordinate edge from parent[dotted]}}
%  				child[onslide={<7,12,17> st_r}]
%  				 {node {1} child foreach \a in {0,1}{coordinate edge from parent[dotted]}}
%  				}
%  			}
%  		}
%  	;
%  	
%  	\path[onslide={<3> hlarrow}] (root-1) .. controls +(3,-0.5) .. (root-2);
%  	\path[onslide={<6-8,11-13> hlarrow}] (root-1-1) .. controls +(1.5,-0.5) .. (root-1-2);
%  	\path[onslide={<7-8,17-> hlarrow}] (root-2-1-1) .. controls +(0.75,-0.5) .. (root-2-1-2);
%  	\path[onslide={<13,18> hlarrow}] (root-2-2-1-1) .. controls +(0.375,-0.5) .. (root-2-2-1-2);
%  	\path[onslide={<8,13> hlarrow}] (root-2-2-2-1-1) .. controls +(0.1875,-0.5) .. (root-2-2-2-1-2);
%  		 
%  	\node[draw=none, font=\large, text=blue]  [above of=root-1] {$\only<5,10>{a}\only<15>{id}$};
%  	\node[draw=none, font=\large, text=green!60!black]  [above of=root-2] {$\only<5>{c}\only<10>{d}\only<15>{b}$};
%  	\node[draw=none, font=\large, text=blue]  [above of=root-2-1] {$\only<6,16>{a}\only<11>{id}$};
%  	\node[draw=none, font=\large, text=green!60!black]  [above of=root-2-2] {$\only<6>{d}\only<11>{b}\only<16>{c}$};
%  	\node[draw=none, font=\large, text=blue]  [above of=root-2-2-1] {$\only<7>{id}\only<12,17>{a}$};
%  	\node[draw=none, font=\large, text=green!60!black]  [above of=root-2-2-2] {$\only<7>{b}\only<12>{c}\only<17>{d}$};
%\end{tikzpicture}
%\end{center}
%
%\begin{overlayarea}{\textwidth}{2em}
%\only<1>{Wir bestimmen die Bilder von $v:=(1,1,-1,1)\in V(T^{(2)})$ bzgl. $a,b,c$ und $d$.}
%\only<2,3>{
%	\begin{equation*}
%	a(1,1,-1,1)\onslide<3>{=(-1,1,-1,1)}
%	\end{equation*}}
%\only<4-8>{\vspace{-2.5em}
%	\begin{align*}
%	b(1,1,-1,1)	 &\onslide<5-8>{=(1,c(1,-1,1))}\onslide<6-8>{=(1,1,d(-1,1))}\onslide<7,8>{=(1,1,-1,id(1))}\\
%				 &\onslide<8>{=(1,1,-1,1)=v}
%	\end{align*}}
%\only<9-13>{\vspace{-2.5em}
%	\begin{align*}
%	c(1,1,-1,1) 	&\onslide<10-13>{=(1,d(1,-1,1))}\onslide<11-13>{=(1,1,b(-1,1))}\onslide<12,13>{=(1,1,-1,a(1))}\\
%				&\onslide<13>{=(1,1,-1,-1)}
%	\end{align*}}
%\only<14-18>{\vspace{-2.5em}
%	\begin{align*}
%	d(1,1,-1,1) 	&\onslide<15-18>{=(1,b(1,-1,1))}\onslide<16-18>{=(1,1,c(-1,1))}\onslide<17,18>{=(1,1,-1,a(1))}\\
%				&\onslide<18>{=(1,1,-1,-1)}
%	\end{align*}}
%\end{overlayarea}
%\end{frame}
%
%%\begin{frame}
%%\begin{exam}
%%F\"ur $v:=(1,1,-1,1)\in V(T^{(2)})$ ist
%%\begin{align*}
%%a(v)&=(-1,1,-1,1),\\
%%b(v)&=(1,c(1,-1,1))=(1,1,d(-1,1))=(1,1,-1,id(1))\\
%%&=(1,1,-1,1)=v,\\
%%c(v)&=(1,d(1,-1,1))=(1,1,b(-1,1))=(1,1,-1,a(1))\\
%%&=(1,1,-1,-1) \text{ und}\\
%%d(v)&=(1,b(1,-1,1))=(1,1,c(-1,1))=(1,1,-1,a(1))\\
%%&=(1,1,-1,-1).
%%\end{align*}
%%\end{exam}
%%\end{frame}
%
%\subsubsection{Unendliche Ordnung von $\Gamma$}
%\begin{frame}
%\begin{theo}
%F\"ur die Erzeuger $a,b,c$ und $d$ von $\Gamma$ gilt
%\begin{enumerate}
%\item \begin{equation*}
%a^2=b^2=c^2=d^2=id
%\end{equation*}
%und
%\item \begin{equation*}
%bc = cb = d,\quad bd = db = c,\quad cd = dc = b.
%\end{equation*}
%\end{enumerate}
%\end{theo}
%\end{frame}
%
%\begin{frame}
%Eine direkte Konsequenz des vorherigen Satzes ist, dass jedes Element von $\Gamma$ von der Form
%\begin{equation*}
%u_0au_1au_2a\ldots u_lau_{l+1}
%\end{equation*}
%ist, wobei $u_1\ldots u_{l}\in\lbrace b,c,d\rbrace$ und $u_0,u_{l+1}\in\lbrace id,b,c,d\rbrace$
%\end{frame}
%
%\begin{frame}
%\begin{defin}
%F\"ur $m\in\Z$, $m\geq 1$ sei $St_\Gamma(m):=St(m)\cap \Gamma$ die \emph{$m$-te Stabilisatorgruppe von $\Gamma$.}
%\end{defin}
%\begin{lem}<2>
%Ein Automorphismus $\gamma=u_0au_1au_2a\ldots u_lau_{l+1}$ ist genau dann in $St_\Gamma(1)$, wenn die Anzahl an Automorphismen $a$ in $\gamma$ gerade ist, d.h. $l$ ist ungerade.
%\end{lem}
%\end{frame}
%
%
%\begin{frame}{$\Gamma$ ist unendlich}
%\begin{lem}
%F\"ur die Automorphismen $a,b,c$ und $d$ gelten
%\begin{align*}
%\psi(aba)&=(c,a)\\
%\psi(aca)&=(d,a)\text{ und}\\
%\psi(ada)&=(b,id).
%\end{align*}
%\end{lem}
%
%\begin{theo}<2>
%Die Abbildung $\psi_1\colon St_\Gamma(1)\to \Gamma$ definiert durch $\gamma\mapsto \gamma\big|_{T_{(1)}^{(2)}}$ ist ein Gruppenepimorphismus.\\
%Da $a\notin St_\Gamma(1)$ folgt, dass $\Gamma$ unendlich ist.
%\end{theo}
%\end{frame}
%
%\subsubsection{Periodizit\"at von $\Gamma$}
%
%\begin{frame}{$\Gamma$ ist periodisch}
%\begin{theo}
%$\Gamma$ ist eine 2-Gruppe, d.h. f\"ur alle $\gamma\in\Gamma$ existiert ein $n\in\Z, n\geq 0$, sodass $\gamma^{2^n}=id$. 
%\end{theo}
%\end{frame}
%
%\section{Weiterf\"uhrendes zu Burnside-Problemen}
%\subsection{Geschichte und Varianten}
%\begin{frame}
%\begin{description}
%\item[1902:] \textsc{William Burnside} stellt in seinem Artikel
%\begin{quotation}
%On an unsettled question in the\\ theory of discontinuous groups
%\end{quotation}
%das \emph{Unbounded Burnside's Problem} auf.\\
%Da er das Problem nicht l\"osen kann, fragt Burnside die "`einfachere"' Frage
%\begin{quotation}
%Ist jede endlich erzeugte Gruppe mit\\ endlichem Exponent endlich?
%\end{quotation}
%(\emph{Bounded Burnside's Problem})
%\end{description}
%\end{frame}
%
%\begin{frame}
%\begin{description}
%\item[um 1930:] Das \emph{Restricted Burnside's Problem} wird formuliert:
%\begin{quotation}
%Gibt es bis auf Isomorphie nur endlich\\
%viele endliche Gruppen, die von \\
%$n$ Elementen mit Ordnung $m$ \\
%erzeugt werden?
%\end{quotation}
%\item[1964:] \textsc{Evgeny Golod} and \textsc{Igor Shafarevich} konstruieren ein Gegenbeispiel f\"ur das \textit{Unbounded Burnside's Problem.}
%\end{description}
%\end{frame}
%
%\begin{frame}
%\begin{description}
%\item[1968:] \textsc{Pyotr Novikov} und \textsc{Sergei Adian} beweisen die Existenz von unendlichen, endlich erzeugten Gruppen mit Exponent $n$ f\"ur alle ungeraden $n\geq 4381$ (\textit{Novikov-Adian-Theorem}).
%\item[1975:] \textsc{Sergei Adian} verallgemeinert die Aussage f\"ur ungerade Exponenten gr\"o\ss{}er 664.
%\item[1980:] \textsc{Rostislav Grigorchuk} konstruiert mit der 1.~Grigorchuk-Gruppe ein zweites Gegenbeispiel f\"ur das \emph{Unbounded Burnside's Problem.}
%\end{description}
%\end{frame}
%
%\begin{frame}
%\begin{description}
%\item[1982:] \textsc{Alexander Yu. Ol\'{s}hanski\u{i}} ver\"offentlicht einen auf geometrischen Argumenten basierenden Beweis des \textit{Novikov-Adian-Theorem} f\"ur ungerade Exponenten gr\"o\ss{}er $10^{10}$.
%\item[1992:] \textsc{Sergei Vasilievich Ivanov} widerlegt das \emph{Bounded Burnside's Problem} f\"ur alle geraden Exponenten $n\geq 2^{48}$, die durch $2^9$ teilbar sind.
%\end{description}
%\end{frame}
%
%\begin{frame}
%\begin{description}
%\item[1994:] \textsc{Efim Zelmanov} beantwortet das \emph{Restricted Burnside's Problem} positiv und erh\"alt f\"ur seine Arbeit die Fields-Medaille.
%\item[1996:] \textsc{I. G. Lys\"{e}nok} gibt Gegenbeispiele des \textit{Bounded Burnside's Problem} f\"ur alle Exponenten $n\geq 8000$ an.
%\end{description}
%\end{frame}
%
%\section*{Literatur}
%\begin{frame}{Literatur}
%
%\begin{thebibliography}{W}
%\bibitem{Hud} \textsc{Hudec, Drew A.}: \emph{On the Burnside Problem}\\
%University of Chicago REU, 2006
%
%\bibitem{Sap} \textsc{Sapir, Mark V.}: \emph{Combinatorial algebra: syntax and semantics}\\
%Springer, 2013
%
%\end{thebibliography}
%
%\end{frame}
%
%\begin{frame}[standout]
%    Thank you for your attention!
%\end{frame}
\end{document}
